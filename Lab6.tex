% Exam Template for UMTYMP and Math Department courses
%
% Using Philip Hirschhorn's exam.cls: http://www-math.mit.edu/~psh/#ExamCls
%
% run pdflatex on a finished exam at least three times to do the grading table on front page.
%
%%%%%%%%%%%%%%%%%%%%%%%%%%%%%%%%%%%%%%%%%%%%%%%%%%%%%%%%%%%%%%%%%%%%%%%%%%%%%%%%%%%%%%%%%%%%%%

% These lines can probably stay unchanged, although you can remove the last
% two packages if you're not making pictures with tikz.
\documentclass[11pt]{exam}
\RequirePackage{amssymb, amsfonts, amsmath, latexsym, verbatim, xspace, setspace}
\RequirePackage{tikz, pgflibraryplotmarks}

% By default LaTeX uses large margins.  This doesn't work well on exams; problems
% end up in the "middle" of the page, reducing the amount of space for students
% to work on them.
\usepackage[margin=1in]{geometry}
%\usepackage{tkz-euclide}
\usepackage{multicol}
\usepackage{graphicx}
\usepackage{tikz,pgfplots}
\usepackage{listings}
\usepackage{pdfpages}
\usepackage{minitoc} %% Required
\usepackage{tabularx}

% Here's where you edit the Class, Exam, Date, etc.
\newcommand{\class}{Calculus I}
\newcommand{\term}{Fall 2020}
\newcommand{\examnum}{Lab 6}
\newcommand{\examdate}{August 20, 2020}

% For an exam, single spacing is most appropriate
\singlespacing
% \onehalfspacing
% \doublespacing

% For an exam, we generally want to turn off paragraph indentation
\parindent 0ex

\begin{document} 
	
	% These commands set up the running header on the top of the exam pages
	\pagestyle{head}
	\firstpageheader{}{}{}
	\runningheader{\class}{\examnum\ - Page \thepage\ of \numpages}{\term}
	\runningheadrule
	
	\begin{flushright}
		\begin{tabular}{p{2.8in} r l}
			\textbf{\class} & \textbf{Name (Print):} & \makebox[2in]{\hrulefill}\\
			\textbf{\term} &&\\
			\textbf{\examnum} &&\\
			%\textbf{\examdate} &&\\
			%\textbf{Due Date: \duedate}
			%\textbf{Time Limit: \timelimit} & Teaching Assistant & \makebox[2in]{\hrulefill}
		\end{tabular}\\
	\end{flushright}
	
Show all your work, cite your sources, and type your answers for full credit.\\

Materials needed: tape measure, scissors, ruler or phone measurement app, lev-o-gage or suncalc phone app, thermometer or Google
	
	\rule[1ex]{\textwidth}{.1pt}
	
	\setlength{\columnsep}{0.5 in}
	
	\begin{questions}
		
		\addpoints
		
		\question[5] You have a bicyclist riding his bike at 2ft./sec on a bridge that is 6ft. tall traveling east.  You have a kayaker paddling under the bridge at 3ft./sec traveling south.  The kayaker arrives directly under the center of the bridge 1 second before the bike reaches the center of the bridge.  They both travel for 4 seconds after the bike passes the center of the bridge.  How fast are they separating?
		
		\question[10] You and a friend are in front of the Science Building.  You head inside for Calculus class and your friend walks to Jazzman’s to grab some coffee.  How quickly is the distance between you and your friend increasing after 5 seconds, assuming that you both walk in straight lines for those 5 seconds?  (Hint:  Law of Cosines)
		
		\question[10] Fire a nerf gun in the air and film the entire flight from 20 ft away.  Let $\theta$ be the angle between your line-of-sight of the nerf dart and the floor.  What is the rate of change of the angle in terms of t?
			
		
		
	\end{questions}
	
\end{document}
