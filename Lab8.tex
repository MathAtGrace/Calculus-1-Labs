% Exam Template for UMTYMP and Math Department courses
%
% Using Philip Hirschhorn's exam.cls: http://www-math.mit.edu/~psh/#ExamCls
%
% run pdflatex on a finished exam at least three times to do the grading table on front page.
%
%%%%%%%%%%%%%%%%%%%%%%%%%%%%%%%%%%%%%%%%%%%%%%%%%%%%%%%%%%%%%%%%%%%%%%%%%%%%%%%%%%%%%%%%%%%%%%

% These lines can probably stay unchanged, although you can remove the last
% two packages if you're not making pictures with tikz.
\documentclass[11pt]{exam}
\RequirePackage{amssymb, amsfonts, amsmath, latexsym, verbatim, xspace, setspace}
\RequirePackage{tikz, pgflibraryplotmarks}

% By default LaTeX uses large margins.  This doesn't work well on exams; problems
% end up in the "middle" of the page, reducing the amount of space for students
% to work on them.
\usepackage[margin=1in]{geometry}
%\usepackage{tkz-euclide}
\usepackage{multicol}
\usepackage{graphicx}
\usepackage{tikz,pgfplots}
\usepackage{listings}
\usepackage{pdfpages}
\usepackage{minitoc} %% Required
\usepackage{tabularx}
\usepackage{hyperref}


% Here's where you edit the Class, Exam, Date, etc.
\newcommand{\class}{Calculus I}
\newcommand{\term}{Fall 2020}
\newcommand{\examnum}{Lab 8}
\newcommand{\examdate}{August 20, 2020}

% For an exam, single spacing is most appropriate
\singlespacing
% \onehalfspacing
% \doublespacing

% For an exam, we generally want to turn off paragraph indentation
\parindent 0ex

\begin{document} 
	
	% These commands set up the running header on the top of the exam pages
	\pagestyle{head}
	\firstpageheader{}{}{}
	\runningheader{\class}{\examnum\ - Page \thepage\ of \numpages}{\term}
	\runningheadrule
	
	\begin{flushright}
		\begin{tabular}{p{2.8in} r l}
			\textbf{\class} & \textbf{Name (Print):} & \makebox[2in]{\hrulefill}\\
			\textbf{\term} &&\\
			\textbf{\examnum} &&\\
			%\textbf{\examdate} &&\\
			%\textbf{Due Date: \duedate}
			%\textbf{Time Limit: \timelimit} & Teaching Assistant & \makebox[2in]{\hrulefill}
		\end{tabular}\\
	\end{flushright}
	
Show all your work, cite your sources, and type your answers for full credit.\\

Materials needed: paper and scotch tape
	
	\rule[1ex]{\textwidth}{.1pt}
	
	\setlength{\columnsep}{0.5 in}
	
	\begin{questions}
		
		\addpoints
		
		\question[15]
		Use hourly data to find the average temperature during the past 24 hours.
		Go to \url{https://w1.weather.gov/data/obhistory/KFWA.html} and enter the temperature each hour from the past 24 hours into a Desmos table.  Then try fitting different a cubic polynomial to the data ($y_1 \sim ax_1^3 + bx_1^2 + cx_1 + d$)
		\begin{parts}
			\part Export a picture of your graph and put it into this lab report.
			\part What is the absolute maximum/minimum temperature during 24 hours from the data?
			\part What is the absolute maximum/minimum temperature estimate using the trendline?
			\part Explain why there is a difference between your answers to the two parts above.
		\end{parts}
		
		\question[5]
		When a critically damped RLC circuit is connected to a voltage source, the current $I$ in the circuit varies with time according to the equation
		\[I = \left(\frac{V}{L}\right)te^{-Rt/(2L)}\]
		where $V$ is the applied voltage, $L$ is the inductance, and $R$ is the resistance (all of which are constant).
		
		Suppose an RLC circuit with a resistance of 30.0 volt/amp and an inductance of 0.400 volt $\cdot$ sec/amp is attached to a 12.0-volt voltage source. Find the maximum current that will occur in the circuit.
			
		\question[5] 2.	A window is being built and the bottom is a rectangle and the top is a semicircle. If there is 12 meters of framing materials what must the dimensions of the window be to let in the most light?
		
		\question[5] Using one sheet of paper, make a triangular box without a lid that maximizes the volume.  This is a competition.  You will get points based on your rank in the class.  You will need to calculate the volume of your box.\\
		
		Rules:
		\begin{itemize}
			\item The base of the boat must be a triangle
			\item You are allowed to cut pieces off of your paper and tape them back on.
		\end{itemize}
		
		
	\end{questions}
	
\end{document}
