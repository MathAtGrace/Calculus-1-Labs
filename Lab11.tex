% Exam Template for UMTYMP and Math Department courses
%
% Using Philip Hirschhorn's exam.cls: http://www-math.mit.edu/~psh/#ExamCls
%
% run pdflatex on a finished exam at least three times to do the grading table on front page.
%
%%%%%%%%%%%%%%%%%%%%%%%%%%%%%%%%%%%%%%%%%%%%%%%%%%%%%%%%%%%%%%%%%%%%%%%%%%%%%%%%%%%%%%%%%%%%%%

% These lines can probably stay unchanged, although you can remove the last
% two packages if you're not making pictures with tikz.
\documentclass[11pt]{exam}
\RequirePackage{amssymb, amsfonts, amsmath, latexsym, verbatim, xspace, setspace}
\RequirePackage{tikz, pgflibraryplotmarks}

% By default LaTeX uses large margins.  This doesn't work well on exams; problems
% end up in the "middle" of the page, reducing the amount of space for students
% to work on them.
\usepackage[margin=1in]{geometry}
%\usepackage{tkz-euclide}
\usepackage{multicol}
\usepackage{graphicx}
\usepackage{tikz,pgfplots}
\usepackage{listings}
\usepackage{pdfpages}
\usepackage{minitoc} %% Required
\usepackage{tabularx}
\usepackage{hyperref}


% Here's where you edit the Class, Exam, Date, etc.
\newcommand{\class}{Calculus I}
\newcommand{\term}{Fall 2020}
\newcommand{\examnum}{Lab 11}
\newcommand{\examdate}{September 18, 2020}

% For an exam, single spacing is most appropriate
\singlespacing
% \onehalfspacing
% \doublespacing

% For an exam, we generally want to turn off paragraph indentation
\parindent 0ex

\begin{document} 
	
	% These commands set up the running header on the top of the exam pages
	\pagestyle{head}
	\firstpageheader{}{}{}
	\runningheader{\class}{\examnum\ - Page \thepage\ of \numpages}{\term}
	\runningheadrule
	
	\begin{flushright}
		\begin{tabular}{p{2.8in} r l}
			\textbf{\class} & \textbf{Name (Print):} & \makebox[2in]{\hrulefill}\\
			\textbf{\examnum} &&\\
			%\textbf{\examdate} &&\\
			%\textbf{Due Date: \duedate}
			%\textbf{Time Limit: \timelimit} & Teaching Assistant & \makebox[2in]{\hrulefill}
		\end{tabular}\\
	\end{flushright}
	
Show all your work, cite your sources, and type your answers for full credit.\\

Materials needed: none
	
	\rule[1ex]{\textwidth}{.1pt}
	
	\setlength{\columnsep}{0.5 in}
	
	\begin{questions}
		
		\addpoints
		
		\question[5] Two poles, one 6 meters tall and one 15 meters tall, are 20 meters apart.  A length of wire is attached to the top of each pole and it is also staked to the ground somewhere between the two poles.  Where should the wire be staked so that the minimum amount of wire is used?
		
		\question[5] Find the absolute maximum for the volume of an open box formed by cutting out squares from the corners of a piece of paper and fold up to form an open box.
		
		\question[5] You accidentally go to the wrong room for Calculus and you need to get to McClain 103 the quickest way possible from SCI 211.  What route do you take?  Why?
		
		\question[5]The electric power P in watts in a direct-current circuit with two resistors $R_1$ and $R_2$ connected in parallel is
		\[
		P = \frac{VR_1 R_2}{(R_1+R_2 )^2}
		\]
		Where $V$ is the voltage.  If $V$ and $R_1$ are held constant, what resistance $R_2$ produces maximum power?
		
		\question[5] A taxi company transfers 400 people each day at \$30 per customer per day.  They estimate that for every \$1 increase results in the loss of 20 customers.  What price should they charge to maximize revenue?  What would the maximum revenue be?
		
		
		
	\end{questions}
	
\end{document}
